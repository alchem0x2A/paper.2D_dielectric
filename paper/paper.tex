% Created 2017-10-05 Thu 16:11
% Intended LaTeX compiler: pdflatex
\documentclass[journal=ancac3,manuscript=article,email=true,hyperref=true,keywords=true]{achemso}
% \usepackage{minted}
% \usepackage{graphicx}
%\usepackage{float}
%\usepackage{xcolor}
%\usepackage{amsmath}
%\usepackage{amssymb}
%\usepackage{fontspec}
%\renewcommand{\figurename}{}

\setkeys{acs}{usetitle = true}
\renewcommand*{\bibnumfmt}[1]{#1.}
\usepackage{lineno}
\linenumbers
\usepackage{lmodern}
\usepackage{amsmath}
\usepackage{graphicx}
\usepackage{hyperref}
\usepackage{caption}
\usepackage[T1]{fontenc}
\renewcommand{\sfdefault}{cmr}


\author{Dale Hughes}
\affiliation{School of Mathematics and Physics, Queen's University Belfast, BT7 1NN, United Kingdom}
\altaffiliation{D. H. and T. T. Contributed equally to this work}

\author{Tian Tian}
\affiliation{Institute for Chemical and Bioengineering, ETH Z{\"{u}}rich,  Vladimir Prelog Weg 1, CH-8093 Z{\"{u}}rich, Switzerland}
\altaffiliation{D. H. and T. T. Contributed equally to this work}

\author{Declan Scullion}
\affiliation{School of Mathematics and Physics, Queen's University Belfast, BT7 1NN, United Kingdom}

\author{Lu Hua Li}
\affiliation{Institute for Frontier Materials, Deakin University, Waurn Ponds, Victoria, Australia}

\author{Chih-Jen Shih}
\affiliation{Institute for Chemical and Bioengineering, ETH Z{\"{u}}rich,  Vladimir Prelog Weg 1, CH-8093 Z{\"{u}}rich, Switzerland}

\author{Jonathan N. Coleman}
\affiliation{School of Physics, Centre for Research on Adaptive Nanostructures and Nanodevices (CRANN) and Advanced Materials and BioEngineering Research (AMBER), Trinity College Dublin, Dublin 2, Ireland.}

\author{Manish Chhowalla}
\affiliation{Materials Science and Engineering, Rutgers University, 607 Taylor Road, Piscataway, New Jersey 08854, USA.}

\author{Elton J. G. Santos}
\email{e.santos@qub.ac.uk}
\affiliation{School of Mathematics and Physics, Queen's University Belfast, BT7 1NN, United Kingdom}
\keywords{two-dimensional materials, dielectric constant, screening, first principles calculation, band gap, binding energy, scaling relationships, Moss relation}
\date{}

\title{Universal Understanding of the Dielectric Constants of Two-Dimensional Materials}
%Universal Understanding of the Dielectric Constants of Two-Dimensional Materials
%Unified Understanding of the Dielectric and Electronic Properties of Two-Dimensional Materials
%Unified Understanding of the Dielectric Constants of Two-Dimensional Materials
\begin{document}

%\newpage{}


\begin{abstract}
Dielectric constant, which defines the polarization of the media, is a key quantity in condensed matter. It describes to a large degree the electron-electron interaction, 
which has a crucial effect on band gaps, optical excitations, and screening. 
Despite its fundamental transcendence, the determination of such quantity 
in two-dimensional materials is a challenging task because 
their intrinsic low-dimensionality hampers accurate measurements.  
Here we show that, for the wide range of known layered crystals, 
a unifying framework for predicting dielectric constants using a 
single universal descriptor can be established. 
A set of scaling relations is found linking dielectric 
constants and band gaps which results in a general 
electronic structure despite of the chemical and physical properties of the layer. 
This points to a Moss-like behavior where novel descriptors 
for photoconductivity and quantum-capacitance appear naturally. 
Our results successfully rationalize the available empirical data and lay the foundation of 
future electronic-dielectric design strategies. 

\end{abstract}


\section{Introduction}
\label{sec:orgb64aacb}

The dielectric constant (also known as the relative permittivity) 
which characterizes the screening of the electromagnetic field within
a medium, plays a crucial role in bridging various fundamental material
properties regarding electron-electron and electron-photon
interactions, such as electronic polarization, polarizability, optical
absorption, and exciton binding energy. 
The central place
of dielectric constant in physics drives the pursuit for a
unified model between electronic and dielectric or optical properties of materials. 
In fact, some pioneering works in the 50's discovered relationships between the dielectric
constants (high-frequency) $\varepsilon$ and other properties, in particular the bulk band gap
$E_g$. Moss\cite{Moss_1950_relation,Moss52book,Moss59book} 
showed that $E_g$ and $\varepsilon$ 
follow a semi-empirical relationship, known as the Moss relation
\cite{Moss_1950_relation,Moss52book,Moss59book,Moss_1985_n_Eg}:

\begin{equation}
\label{eq:Moss-relations}
\varepsilon^{2} E_{\mathrm{g}} = 95~ eV
\end{equation}
which can be related to the high-frequency refractive index $n_{0}$, via $\varepsilon=n_{0}^{2}$,   
and the long wavelength threshold $\lambda_{c}$ which determines the photoconductivity properties\cite{Moss_1950_relation}:

\begin{equation}
\label{eq:Moss-wave}
n_{0}^{4} /\lambda_{c} = 77~ \mu m^{-1}
\end{equation}

Other similar
semi-empirical models for the relationship between \(\varepsilon\) and
\(E_{\mathrm{g}}\) were also proposed
\cite{Ravindra_1979_eps_Eg,Ravindra_1980_model,Ravindra_2007_Eg_rev}. 
Despite the minute difference between the formulae used, it is clear that the
there exist certain general electronic relations in bulk
semiconductors, which are scaled by the dielectric constant and further
account for the variation in band gap between different
materials. The existence of such general relationship is also of high
practical importance: it is generally more straightforward to measure
the band gap \(E_{\mathrm{g}}\) than the dielectric constant \(\varepsilon\),
as the latter is strongly dependent on several critical 
details on the sample measurements, such as contamination, disorder, 
and model interpretation. With the Moss
relationship or other similar models, it is feasible to predict the
dielectric constant of a bulk semiconductor with a reasonable degree
of certainty starting from the magnitude of its band gap, 
as well as identify whether a compound is photoconductor. 
Indeed, the usage of such relations has spread out across 
the semiconductor community for more than 60 years. 

Two-dimensional (2D) materials, the atomically-thin crystalline
layers, are known to have distinct electronic properties from their
bulk counterparts
\cite{Ando_1982_electron_2D,Stern_1967_polarizability_2DEG}. One
important feature of 2D materials is that the electric screening is
attenuated and anisotropic\cite{Das-Sarma:2011aa,Ando_1982_electron_2D}. 
The empirical
relations of dielectric constant in bulk materials, like the Moss
relation (Eq.\ref{eq:Moss-relations}), is thus invalid within the context of 2D materials, and has
to be modified. The quest for a general understanding of the
dielectric constant of 2D materials has even greater importance than
the bulk materials: the dielectric constant of a 2D material is
highly influenced by its surrounding environment
\cite{Keldysh_1979_eps_multi,Trolle_2017_eps_subst} and it has caused
considerable debate experimentally\cite{Raja:2017aa,Chernikov_2014_EB_MoS2_2D3D}. 
On the contrary, other
optoelectronic properties such as the band gap and binding energy can
be experimentally more easily assessed. 
The lack of both sufficient
data bank and sound experimental results requires the force from
computational chemistry to discover the underlying relation between
the dielectric constant and other properties of 2D materials. 

Here we show that the dielectric constant provides a unique opportunity for 
the development of universal scaling relationships between electronic and 
dielectric properties of different 2D materials. The reduced screening features 
allow to observe the true nature of the electronic interactions in 
the 2D-world, which indicates a universal behavior despite the chemical 
and physical composition of the material. 
We explicitly obtained scaling 
relationships between dielectric constant and several 
other quantities, such as band gaps, and exciton binding 
energies, which describes a vast library of layered materials. 
This fundamental understanding can be further explored
looking for a new descriptor of photoconductivity, 
where 2D-photoconductors follow a similar pattern on refractive index 
and absorption edge.    
Moreover, using multiscale methods we unveil the close relation between quantum-capacitance and dielectric 
constants, which provides a map of the screening properties in terms of orbital decompositions.  
This framework is also used to successfully unify the concept of dielectric constant 
at different dimensionalities as its 2D-magnitude is obtained 
throughout the knowledge of few details of the 
3D-bulk electronic structure. The discovery of scaling relationships 
between electronic and dielectric properties in 
2D compounds introduces a new and general pathway for the exploration of their complex 
physical and chemical phenomena, and ultimately pushes the boundary 
of the understanding of electronic screening in two-dimensions. 

\section{Results}
\label{sec:org93965d9}

\subsubsection{2D Moss relations}
We study several types of 2D materials of different electronic and optical properties, 
including transition metal dichalcogenides (TMDCs, with the formula MX\(_{\text{2}}\),
where M is a metal in group 4, 6, 10 and X=O, S, Se, Te), metal
monochalcogenides (GaS, GaSe), cadmium halides (CdX$_2$, X=Cl, I), 
hexagonal boron nitride (hBN), graphene derivatives (fluorographene (f-graphene), graphane) 
thin layers of organic-inorganic perovskites (CH$_3$NH$_3$PbBr$_3$),  
and phosphorene as displayed in Figure \ref{fig-1}{\bf a}. For the TMDCs, we consider 
2 lattice phases, namely the 2H (P\(\bar{6}\)m2 space group) and 1T (P3m1 space group). 
The elements contained in the materials involved in our 
simulations are summarized in the periodic table in Fig. \ref{fig-1}{\bf b}, 
which are grouped accordingly to their dielectric-electronic scaling 
relationships as discussed below. 
%%%Write here PBE data %%%%%%%
We calculate both the high-frequency dielectric constant tensor  
$\varepsilon_{\alpha,\beta}$ 
and band gaps $E_{g}$ at the level of 
hybrid density functional theory (DFT) using the Heyd-Scuseria-Ernzerhof
hybrid functionals (HSE06) including spin-orbit 
coupling to avoid limitations on the description of band gaps 
and dielectric constants (see {\it Methods} for details). 
We have also utilized different 
functionals (e.g. PBE) and approximations, 
such as many-body $G_0W_0$, which resulted in similar conclusions. 
See Supplementary Note 1 and Supplementary Figures 1-3 for details. 


The in-plane $\varepsilon_{\parallel}$ and out-of-plane $\varepsilon_{\perp}$ 
components of the dielectric tensor as functions of E$_{g}$
are plotted in Figure \ref{fig-2}{\bf a-b}, respectively. 
For the $\varepsilon_{\parallel}-E_{g}$ relation, we notice two distinguished 
behaviors: a linear (green circles) and a power law (orange squares) 
variations of $\varepsilon_{\parallel}$ with $E_{g}$. Interestingly, such division 
is also element-dependent: the
TMDCs of group 4 and 14, hBN, graphene derivatives, and phosphorene
belong to the power group, while TMDCs of group 6 and 10, GaS, GaSe,
and CdI\(_{\text{2}}\) following the linear law (see Fig. \ref{fig-1}{\bf b}). 
The best fitting of the power group gives the relation: 

\begin{equation}
\label{Moss-2D-power}
\varepsilon_{\parallel}=3.68/E_{g}^{0.50}
\end{equation}
or in term of the refractive index and long-wavelength:

\begin{equation}
\label{Moss-2D-power-lambda}
n_{0}^{4}/\lambda_{c} = 10.92~ \mu m^{-1}
\end{equation}

The linear group is well fitted with: 

\begin{equation}
\label{Moss-2D-linear}
\varepsilon_{\parallel}=7.50-1.37{\rm E}_{g} 
\end{equation}

We notice that the power law of 2D layers (Eq. \ref{Moss-2D-power}) 
resembles the 3D Moss relation (Eq. \ref{eq:Moss-relations})
for bulk materials, however with a much smaller constant, 
as seen in the comparison in Fig. \ref{fig-2}{\bf a} using Eq. \ref{eq:Moss-relations}. 
The linear law in turn 
differs from the original Moss relation, but it is close to other approaches \cite{Ravindra_1979_eps_Eg,Ravindra_1980_model,Ravindra_2007_Eg_rev}.
Conversely, $\varepsilon_{\perp}-E_{g}$ relations show much less
variation than the in-plane component: most of the values of 
$\varepsilon_{\perp}$ are within 1.2 and 1.6, indicating
that the dielectric screening perpendicular to the 2D plane is greatly
attenuated. The scattered dots of the power group and the linear group
overlap with each other in Figure \ref{fig-2}{\bf b}, showing that
the $\varepsilon_{\perp}$ has almost no material dependence in terms of the band gap. 
Therefore, it is accurate to state that out-of-plane screening excitations in a monolayer 
would behave similarly despite of its electronic and optical structures.   
Moreover, our \emph{ab initio} calculation results show that the in-plane dielectric
constant of a 2D material is clearly related with its band gap through scaling relations, 
either linear or non-linear (power-law). %maybe a new paragraph %
It is worth mentioning very few experimental reports on either $\varepsilon_{\perp}$ or 
$\varepsilon_{\parallel}$ for 
monolayer materials\cite{BN-epsilon,Mos2-epsilon,In2Se3-epsilon}. 
Nevertheless, the available data 
follows our scaling relations as displayed in Fig. \ref{fig-2}{\bf a}. 


The relations between $\varepsilon$ and E$_{g}$
distinguishes from the 3D Moss equation and other approaches\cite{Ravindra_1979_eps_Eg,Ravindra_1980_model,Ravindra_2007_Eg_rev},  
which indicates that the underlying mechanism in 2D
materials may be different from their bulk counterparts. 
Nevertheless, the possibility to use universal scaling relationships
that accurately describe a vast library of layered compounds using 
a single parameter have not been previously demonstrated. We postulate 
that this is possible because of the reduced screening observed in monolayers, and,
therefore, the dielectric constant does not depend sensibly 
on the material geometry. 
To verify the precision of the $\varepsilon-E_{g}$ relations
discovered, we compare the dielectric constants computed from the
linear and power laws using only the \(E_{g}\) values as input, compared with
the \(\varepsilon\) obtained by time-consuming HSE06 simulations. 
Figure \ref{fig-3}{\bf a} demonstrates the excellent agreement between our 
model's predictions and calculated HSE06 \(\varepsilon_{\parallel}\) across a wide variety of 
2D crystals. The linear correlation of the two sets of
data show a slope of 0.9976, indicating that the model can 
represent well the relation between \(\varepsilon_{\parallel}\) and
\(E_{g}\) by a numerical error smaller than 0.2\%. 
We perform a similar analysis for the out-of-plane
dielectric constant $\varepsilon_{\perp}$ in Figure \ref{fig-3}{\bf b}. 
We notice that the \(\varepsilon_{\perp}\) calculated
using the model has larger deviation from the \(y=x\) line than
\(\varepsilon_{\parallel}\). However, the scale utilized corresponds to magnitudes 
in the range of 1.10 to 1.40, which is 20 times smaller than that at in-plane component. 
This indicates that the model's prediction for \(\varepsilon_{\parallel}\) would give magnitudes within 0.05-0.12 differences relative to HSE06 calculations, 
which is in the numerical accuracy of the simulations. 
The fact that \(\varepsilon_{\perp}\) has
less dependency on \(E_{g}\) clearly distinguishes from the 3D
Moss relation, as a result of the attenuated out-of-plane dielectric
screening in 2D materials. 
Considering this relatively small dispersion
of the out of plane component, we may
conclude that the in-plane dielectric constant of 2D materials has
much greater impact on other properties as normally assumed in 
optical measurements\cite{Chernikov_2014_EB_MoS2_2D3D}. 
For the rest of the discussions we
will focus on the relation between \(\varepsilon_{\parallel}\) with other
2D properties.
It should be noted that simulations carried out at the level of PBE functional resulted in considerably larger 
differences between the model predictions and calculated \(\varepsilon_{\parallel}\) and \(\varepsilon_{\perp}\) (see Supplementary Figure 3). Deviations from either a purely linear fitting without a finite cumulative factor at the model, or 
magnitudes that differ by more than $\sim$24\% between simulations 
and scaling relations are observed. 
%We have also compared our HSE06 simulation results against 
%more robust but computationally involved methods, such as many-body 
%Green's function method ($G_0W_0$) and 
%Bethe-Salpeter (BSE) equation as show in Supplementary Figure 4. We observe that 
%some $G_0W_0$ band gaps are overestimated  relative to the HSE06 ones as they do not  
%not include the electron-hole corrections. Such corrections are included at the level of 
%$G_0W_0$-BSE which shows a sound agreement with the HSE06 magnitudes. This 
%fully justifies the choice of an accurate approximation such hybrids in the 
%description of the electronic-dielectric interplay in 2D materials.  

%The model obtained 
%using such values give substantial differences from the 
%calculated PBE data (see (Figure S2 in the {\it Supplementary Information})). 
%%%

\subsubsection{A descriptor for photoconductivity in two-dimensional materials}

The generality of the Moss' relation was initially used to identify compounds that 
could develop photoconductivity\cite{Moss_1950_relation,Moss-book1}. 
One of the key reasons is the rough constant threshold of Eq.\ref{eq:Moss-wave} for a vast 
library of 3D-bulk materials. This indicates that materials that are photoconductors assume 
similar pattern over band gaps and dielectric constants. 
In practice the photoconductivity depends on several factors, such as optical absorption coefficient, 
mean-free-time, carrier mobility, and thickness. Figure \ref{fig-3}{\bf c} shows, however, 
that a comparable argument as that used by Moss 
holds for 2D materials in terms of  
the behavior of $n_{0}^{4}/\lambda_{c}$ over the entire dataset. 
We have not separated the materials where photoconductivity has been observed 
experimentally from those that would need further study. This incidentally could give different 
line threshold not shown here. It is clear that materials that are photoconductors display 
similar electronic-dielectric properties via close magnitudes of $ \langle n_{0}^{4}/\lambda_{c} \rangle$ 
as initially suggested by Moss\cite{Moss_1950_relation,Moss59book,Moss52book,Moss_1985_n_Eg}. 
On the power-law group, compounds like 2H-SnS$_2$, 
2H-HfS$_2$, 1T-HfS$_2$, 1T-SnSe$_2$, 1T-SnS$_2$, and phosphorene 
have been observed photo-conductivity response, even though different groups 
have reported distinct magnitudes\cite{photo-2Dmaterials,Andres15}. 
Similarly as observed in the 
linear-law compounds, the average value $ \langle n_{0}^{4}/\lambda_{c} \rangle=31.22~\mu m^{-1}$
is larger than that for power-law materials but most of the photoconductors 
assume close magnitudes of $n_{0}^{4}/\lambda_{c}$. Several layered systems 
shown in Fig. \ref{fig-3}{\bf c} are still to be explored experimentally 
to give a thorough picture of the photoconductivity in 2D. However, 
this simple relation between refractive index and 
long wavelength seems to give reliable power-prediction 
on potential layers that could develop such 
features (see Supplementary Note 2 and Supplementary Figure 5 for additional discussions). 


%%%%%Tian you may start writing from here%%%%%%%%%


%%%%%%%%%%%

\subsubsection{Correlating quantum-capacitance and dielectric constants}

This multiscale approach can also be used to show that other macroscopic properties are directly
correlated with the dielectric constant, such as the quantum-capacitance. 
Similar arguments can be used to link dielectric constants at different dimensionalities, such as between 2D-monolayers and their 3D-bulk counterparts, as well as obtain relations between dielectric constants and details of the electronic structure and exciton biding energies. See Supplementary Notes 3-4 and Supplementary Figures 6-8 for detailed discussions.
The random phase approximation (RPA) describes the imaginary dielectric function \(\varepsilon_{2}\)
at long wavelength limit as \cite{Slyom_2008_fundBook,Dressel:2002aa}:

\begin{equation}
\label{eq:RPA-eps2}
\epsilon_{2}(\omega) = \frac{1}{4 \pi \epsilon_0} (\frac{2\pi e}{m_{e} \omega})^{2} \sum_{c, v} J_{cv}|P_{cv}|^{2}
%\varepsilon_{2}(\omega) = (\frac{2 \pi e}{m \omega})^{2} 
%                     \sum_{l,l'} J_{l,l'}(\omega) < l | \mathbf{\mathit{p}} | l' >^{2}
\end{equation}
where $m_e$ is the electron mass, \(\omega\) is frequency, \(c\) and \(v\)
denote the conduction and valence states, \(J_{cv}\) is the joint density
of states (JDOS) and $P_{cv}$ is the dipole operator. The
difference between the \(\varepsilon-E_{\mathrm{g}}\) relations actually
comes from the forms of JDOS in 2D and 3D systems \cite{Ha_2011_introductry}. 
For a 3D semiconductor, the JDOS is energy-dependent:

\begin{equation}
\label{eq:JDOS-3D}
J^{\mathrm{3D}}(\omega) = \frac{1}{2\pi^{2}} (\frac{2 \mu}{ \hbar ^{2}})^{\frac{3}{2}}
                        (\hbar \omega - E_{\mathrm{g}})^{\frac{1}{2}}
\end{equation}
while on strictly 2D, JDOS is a step function and independent of \(\omega\):

\begin{equation}
\label{eq:JDOS-2D}
J^{\mathrm{2D}}(\omega) = 
                        \begin{cases}
                        \displaystyle \frac{1}{\pi} \frac{\mu}{\hbar^{2}} & \omega > E_{\mathrm{g}} \\
                        0                                   & 0 < \omega < E_{\mathrm{g}}
                        \end{cases}
\end{equation}

The real dielectric function can be derived using the Kramers-Kronig
relation \cite{Adachi_1987_dielGaP}:

\begin{equation}
\label{eq:KKR}
\varepsilon_{1}(\omega) = 1 + \frac{2}{\pi} \int_{0}^{\infty}
                     \frac{\omega' \varepsilon_{2}(\omega')}{\omega'^{2} - \omega^{2}} d\omega'
\end{equation}
in the \(\omega \to 0\) limit, and assuming that the dipole transition
matrix $P_{cv}$ is constant we have:

\begin{equation}
\label{eq:eps1-3D}
\varepsilon_1^{\mathrm{3D}}(\omega \to 0) = 1 + \frac{e^2 (2 \mu) ^{\frac{3}{2}}}{8\pi \epsilon_0 m_{e}^{2} \hbar^{2}}E_{\mathrm{g}}^{-\frac{3}{2}}P_{cv}^{2}
%\varepsilon_{1}^{\mathrm{2D}}(\omega \to 0) = 1 + \frac{e^{2} (2 \mu)^{\frac{3}{2}}}{2 m^{2} \hbar}
%                                  E_{\mathrm{g}}^{-\frac{3}{2}} P^{2}
\end{equation}
and
\begin{equation}
  \label{eq:Dressel-eps1}
  \begin{array}{lll}
  \varepsilon_{1}^{\mathrm{2D}}(\omega \to 0) &= \displaystyle 1 + \frac{2}{\pi} \frac{1}{4 \pi \varepsilon_{0}}
                         \frac{4 \pi^2  e^2}{m_{e}^{2}}
                         \frac{\mu}{\pi \hbar^2} P_{cv}^2
                         \int_{E_{g}/\hbar}^{\infty}
                      \frac{1}{\omega'(\omega'^{2} - \omega^{2})} \mathrm{d} \omega' \\
                    &= \displaystyle 1 + \frac{1}{4 \pi \epsilon_{0}} \frac{8 e^2 \pi}{m_{e}^{2}}
                      \frac{\mu}{\pi \hbar^{2}} \frac{1}{2 \omega^{2}} P_{cv}^{2}
                      \ln(1 - \frac{\omega^{2}}{\omega'^{2}}) \Big|_{E_{g}/\hbar}^{\infty} \\
                    &\approx \displaystyle 1 + \frac{1}{4 \pi \varepsilon_{0}}
                      \frac{4e^2 \pi}{m_{e}^2} \frac{\mu}{\pi \hbar^2}
                      \frac{\hbar^2 \omega^2}{\omega^2 E_{g}^2} P_{cv}^2\\
                    &= \displaystyle 1 + \frac{1}{4 \pi \varepsilon_{0}}
                      \frac{4 e^2 \pi \hbar^2}{m_e^2 E_g^2} J_{cv} P_{cv}^{2}
  \end{array}
\end{equation}
\\
It shows that, unlike in the 3D system, the 2D dielectric constant
has direct relation to JDOS and \(E_{\mathrm{g}}\). The JDOS in a free-electron model can be
written as the combined effect of the density of states (DOS) at both
conduction and valence bands: $J(E)^{-n} = \mathrm{DOS}_{\mathrm{C}}(E- E_{\mathrm{g}})^{-n} + \mathrm{DOS}_{\mathrm{V}}(E- E_{\mathrm{g}})^{-n}$, where n=2/3 for 3D case and n=1 for 2D case. 
Note that DOS is essentially the
quantum capacitance (\(C_{\mathrm{Q}}\)) of a material:
\(C_{\mathrm{Q}}(E) = \mathrm{DOS}(E) e^{2}\). Indeed, it is natural to look
into the relation between \(C_{\mathrm{Q}}\) and \(\varepsilon\), which is
also a characteristic of a material's capacitance. 

In order to extract the \(C_{\mathrm{Q}}\) values, we calculated the DOS of each 2D
material by averaging the values at a band edge charge cutoff of 5 x10$^{13}$ $e/cm^2$ (see {\it Methods} for details). 
We first plot \(\varepsilon_{\parallel}\) as a function of \(C_{\mathrm{Q}}\) in the CB for 
various MX$_2$ materials studied here, as shown in Figure \ref{fig-5}{\bf a}.  Although the   %%%%%%%%%%%%%%%%%%
dispersion seems too wide to get a universal relation when considering
all the data together, we find that when categorizing the data by the
metal element group and lattice type, each $\varepsilon - C_Q$ dataset seems to be following linear
relations. We therefore performed linear fitting via $y=ax+b$
for each group of 2D
materials, and plotted the prediction margin as stripe shaders in
Fig. \ref{fig-5}{\bf a}. By varying $b$, we also show the 
prediction margin of each group as stripe shader. 
The fitting parameters $a$, $b$, the coefficient of determination $R^2$ 
and the half-span of prediction margin $\Delta b$, can be found in Supplementary Table 2. 
   
%Interestingly, although the coefficient of determination
%vary among the groups (group 14-1T and group 14-2T show \(R^{2}\) higher
%than 0.95, while group 4-2H and group 6-2H show \(R^{2}\) less than
%0.8), they all show a similar intercept of ca. 1.70.
%%\begin{center}
%\fbox{
%\begin{minipage}[c]{.8\linewidth}
%\textbf{\textsf{\textsc{TODO}}} Add the data sheet in SI
%
%\end{minipage}
%}
%\end{center}
When \(C_{\mathrm{Q}}\) decays to 0 the 2D materials display an universal dielectric
constant. The discrepancy between the value of the intercept may
also be caused by the existence of long-range Coulombic interactions
in the calculations, similar to the phenomenon seen in Figure
\ref{fig-4}{\bf a}. The slope of the \(\varepsilon-C_{\mathrm{Q}}\) curve in each
group indicates that the materials with metals of the same group, have
similar values of dipole transition matrix \(P\). 
Note that in Eq.\ref{eq:Dressel-eps1}, \(E_{\mathrm{g}}\) also serves as a variable apart from
JDOS, we therefore compare the relation between \(\varepsilon_{\parallel}\)
and \(C_{\mathrm{Q}}/E_{\mathrm{g}}^{2}\), as shown in Figure
\ref{fig-5}{\bf b}. Better regression results can be observed for almost all
groups compared with the \(\varepsilon_{\parallel}-C_{\mathrm{Q}}\)
relations. The slopes are also higher than the corresponding groups in
the \(\varepsilon-C_{\mathrm{Q}}\) relations, since the majority of the
materials we studied has \(E_{\mathrm{g}} > 1\) eV. The better
regression results using \(C_{\mathrm{Q}}/E_{\mathrm{g}}^{2}\) indicates
the validity of Eq.\ref{eq:Dressel-eps1} in predicting the 2D dielectric constant. 
Unlike the relation we found for
\(\varepsilon-E_{\mathrm{g}}\), \(\varepsilon-\alpha\) and
\(\varepsilon-E_{\mathrm{b}}\), a universal scaling relation between
\(\varepsilon\) and \(C_{\mathrm{Q}}\) or \(C_{\mathrm{Q}}/E_{\mathrm{g}}^{2}\)
does not seem to exist, instead the relation is found to be more
element-specific. We have also examined the \(\varepsilon\) as a function
of \(C_{\mathrm{Q}}\) or \(C_{\mathrm{Q}}/E_{\mathrm{g}}^{2}\) in the VB,
as shown in Supplementary Figure 9. However a similar scaling relation seems 
inexistent. Such difference between the \(C_{\mathrm{Q}}\) in CB and VB
may imply that the unoccupied orbitals have more influence on the
dielectric constant of 2D materials. 

%%%%%%%%%%Include tian's text herein 
Note that most the materials studied here have similar geometry, such
similarity between the $\varepsilon$ and $C_{\mathrm{Q}}$ (or
$C_{\mathrm{Q}}/E_{\mathrm{g}}^{2}$) may be explained by the orbitals
close to the Fermi level. For instance, comparing TMDCs with metal
atoms from different groups, the change of orbital near the Fermi
level is mainly due to shift of $d$-orbitals, while the $\sigma$ and
$\sigma^*$ orbitals with mainly $s$ and $p$ components, show relatively
less change\cite{Liang84,Kuc14}. Figure \ref{fig-5}\textbf{c} show the
relative shift of orbitals in group 4, 6 and 10 (semiconducting)
TMDCs. The $d_{z^2}$ orbital which has a sharp increase of DOS and
contributes mainly to the CB in group 4, shifts below the Fermi
level in group 6, with the $d_{xy}$ and $d_{x^2-y^2}$ orbitals
with a slower DOS increment as the mainly-contributing orbital in the
CB. When the $d$-orbital occupation number further increases in group
10, all $d$-orbitals shift below the Fermi level, leaving the
$\sigma^*$ orbital as the major component of the CB. The shift of the
$d$-orbital due to increasing occupation has direct effect on the DOS
shape of CB. As seen in Figure \ref{fig-5}\textbf{d}, the two group 6
TMDCs (MoS$_{2}$ and WS$_{2}$) show a plateau with relative low DOS of
0.1 states/eV$^{3}$ at the CB edge, while the group 4 (ZrS$_{2}$
and HfS$_{2}$) show a much higher DOS plateau of $\sim$0.5 states/eV$^3$. 
On the other hand, all materials show similar DOS
shapes at the VB edge. Therefore, the apparent distinction between the states at the
CB edge can be explained by the shift of $d$-orbital as stated above, which is 
also consistent with the fact that the slope of regression between
$\varepsilon$ and $C_{\mathrm{Q}}$ in group 6 is much larger than that in
group 4. The atomic orbital analysis provides a good explanation for grouped
behavior of the $\varepsilon$-$C_{\mathrm{Q}}$ regression relations, and
provides more insights into the origin of the dielectric constant of
2D materials.


%%%Elton's comments%%%%%%%%%%
%Performing an orbital-based analysis on the states responsible 
%for the dielectric screening we observe that most 
%of the $d$-states for group 4, 6 and 10 shift down below the Fermi level with increasing 
%of the occupation number as displayed schematic in Figure \ref{fig-5}{\bf c}. 
%The higher DOS at CB for group 4 may be contributed by 
%the $dz^2$ orbitals; on the other hand the low CB DOS 
%of group 6 and group 10 may be contributed by other 
%orbitals with smoother edges. A comparison between group 4 and 6 in terms of a normalized 
%DOS by $E_g^2$ for illustrative materials (Figure \ref{fig-5}{\bf d}) 
%reveals the main features at CB for both groups. 
%Indeed, group 4 has a more continuous CB edge that extends 
%for several tenths of meV's and larger DOS. This can be translated 
%in higher magnitudes of \(C_{\mathrm{Q}}\) 
%over the accumulated charge density $\sigma (E)$ (see {\it Methods} for details), and therefore 
%less abrupt variation of $\varepsilon_{\parallel}$ as 
%function of \(C_{\mathrm{Q}}/E_{\mathrm{g}}^{2}\) (Fig. \ref{fig-5}{\bf b}).  
%Group 6 in its turn shows a step-edge with a small DOS relative to group 4 at the same energy range, which consequently imply low magnitudes of \(C_{\mathrm{Q}}\) and a sudden dependence of 
%$\varepsilon_{\parallel}$. Group 10 seems to follow a similar description not show here. 
%Overall this analysis suggests a close correlation between DOS and $\varepsilon_{\parallel}$ 
%throughout the quantum capacitance, which seems not established so far for 2D-materials. 

%%%%%%%%%%


\section{Discussion}
\label{sec:org621792b}

In summary, our findings reveal a fundamental knowledge of the close link between 
dielectric, optical and electronic properties of 2D materials. The scaling relationships found
between dielectric constants and several other important quantities, such band gaps, 
excitonic binding energies, and quantum capacitance imply that such correlation is unique 
in the 2D-world. Physical magnitudes apparently with a no 
clear interdependence in 3D-bulk display a strong connection, which seems to be universal 
to different types of 2D crystals. One of the main driving 
forces for such unusual behavior is the reduced screening 
features present in layered materials, which is not present 
in bulk compounds. 

We have shown that a 2D-analogous of the famous Moss' 
relation between band gaps and 
dielectric constants indeed exists with a similar functional 
form but different energy constants. 
Different compounds can also display different scaling relationships, 
which are accurate enough to reproduce computationally 
expensive hybrid HSE06 calculations at precisions 
near 99\% of the simulations just using a single descriptor. 
This allows to use efficient scaling schemes to quickly explore untouched physical
properties (e.g. photoconductivity) searching for new descriptors 
that could help in their identification. 

An in-depth analysis of the dielectric function within the random phase approximation allowed to 
create a direct relation between quantum capacitance and dielectric constants 
using band gaps as a scaling parameter. Several materials throughout different 
groups of the periodic table follow a systematic pattern over the orbital states near the Fermi level, 
which explains most of the features of the model. A direct inter-relation between 
dielectric constants at 2D and 3D bulk is found based on the Lindhard formalism. 
This equation is used to promptly determine the dielectric constant of any 
2D material using simple arguments from the band structures of its bulk 
counterpart. This is a clear step towards new methods of materials screening using bulk properties. 
Our findings pave the way to understand and engineer 
the electronic and dielectric properties of a large library of vdW materials 
for different technological applications in an unified and unprecedented way. 


%The computational screening performed at the level of HSE06 functional is an instrumental tool 
%for a reliable description of band gaps and dielectric constants, and consequently crossovers between physical variables. 
% 
%
%The methods presented herein are meant to quickly determine the 
%dielectric constant of any 2D-material using simple scaling relationships based 
%on the knowledge of the band gap. 




\section{Methods}
\label{sec:org3881bef}

\subsubsection{Density functional theory calculations}
All first-principles density functional theory (DFT) calculations 
were performed using the VASP code\cite{Kresse_1993,Kresse_1996_1,Kresse_1996_2}. 
Projector Augmented Wave (PAW) approach\cite{Kresse_1999_pseudopotentials,PAW-1} 
using the new optimized GW potentials were utilized. Electronic and dielectric properties 
were initially calculated using generalized gradient approximation 
using PBE functional\cite{PBE}. The self-consistent field (SCF) convergence was set to 1.0x10$^{-8}$ eV. We also took into account a fractional component of the exact
exchange $E_x$ from the Hartree-Fock (HF) theory hybridized with the DFT 
exchange-correlation functional at the level of the range-separated HSE06 hybrid
functional\cite{HSE03,HSE06}. The exchange-correlation energy is given by: 

\begin{equation}
E_{xc}^{HSE}=\alpha E_{x}^{HF, SR} (\omega) + (1-\alpha) E_{x}^{\omega PBE, SR}(\omega) + E_{x}^{\omega PBE, LR} (\omega) + E_c^{PBE}
\end{equation}
where $E_{x}^{HF, SR}(\omega)$ is the short-range (SR) HF exchange; $E_{x}^{\omega PBE, SR}(\omega)$ and $E_{x}^{\omega PBE, LR}(\omega)$ are the short and long range (LR) components of the PBE 
exchange functional, respectively; $\omega= 0.20~\AA^{-1}$ is the screening parameter, 
which defines the separation of the SR and LR exchange energies, and $\alpha$ is the 
HF mixing factor that controls the amount of exact Fock exchange energy 
in the functional.\cite{HSE03,HSE06} Note that HSE functional with $\alpha =0$ becomes the
PBE functional.\cite{PBE} Therefore, any limitation of the exchange and correlation functional in the chemical and physical description of the two-dimensional systems 
is improved. It is important to point out that 
such approach is considerably more expensive than 
the PBE functional, which makes our dataset valuable for 
further modeling analysis (see Supplementary Table 3). 
Spin-orbit coupling (SOC) was explicitly included in the calculations. 

The geometries were converged both in cell
parameters and ionic positions, with forces below 0.001~eV/$\AA$. 
A vacuum spacing of at least 25 $\AA$ was used for each material. 
A {\bf k}-point grid of \(7\times7\times1\) was used to relax, with an initial relaxation
carried out at PBE level and a subsequent relaxation carried out at
HSE06, allowing both cell parameters and ionic positions to
relax each time. In VASP, the tag {\sc PREC=High} was used, giving a plane
wave kinetic energy cutoff of 30\% greater than the highest given in
the pseudopotentials used in each material, guaranteeing that absolute
energies were converged to a few meV. A default plane-wave cutoff of 800 eV 
was used in all simulations. A final HSE06 calculation with SOC was then carried out
using a \(21\times21\times1\) {\bf k}-point in the unit cell for each system. 

Density functional perturbation theory (DFPT) is used to calculate the ion clamped
static dielectric tensor showed in the manuscript as\cite{Gajdos:2006aa}:
 
\begin{equation}  
\varepsilon_{\infty}=1/\varepsilon^{-1}_{\alpha,\beta}(0,0) \equiv 1/\lim_{q \to 0}[\varepsilon^{-1} (\mathbf{q},\mathbf{q+G})]
\end{equation}
with $\mathbf{G}=0$. Off-diagonal components of $\varepsilon_{\alpha, \beta}$ ($\alpha \neq \beta$) generally assume 
negligible magnitudes, with the diagonal components $\varepsilon_{\alpha, \beta}$ ($\alpha = \beta$) being 
similar in-plane, e.g. $xx=yy$. 
We assume that the ions do not relax in the
external electric field at optical frequencies and hence the ionic
contribution to the dielectric constant is negligible. 
In fact, this is the case for 2D-materials as recently found\cite{relax-epsilon}. 
Local field effect corrections are included at the exchange-correlation 
potential $V_{xc}$ at either PBE or HSE06 levels. 

%\subsubsection{Many-body $G_0W_0$ and $G_0W_0-BSE$ calculations}
%
%To benchmark the calculations of band gaps using the HSE06 functional, 
%we have performed simulations at the level of many-body $G_0W_0$ approximations 
%plus Bethe-Salpeter equation (BSE). 10 occupied and 86 unoccupied bands were included in the 
%calculations with SOC included.  


%\begin{equation}
%\varepsilon = 1-\upsilon_{c} \chi_{0} (1-f_{xc} \chi_{0})^{-1}
%\end{equation}
%where $\upsilon_{c}$ is the 
%
%$f_{xc}$ is the functional derivative of $V_{xc}$ with respect to the electron density as: 
%
%\begin{equation}
%f_{xc}(1,2)=\frac{\delta V_{xc}(1)}{\delta \rho(2)}
%\end{equation}



\subsubsection{Quantum capacitance}
For quantum capacitance analysis, the quantum capacitance
$C_{\mathrm{Q}}(E)$ at certain energy level $E$ is calculated using
the relation $C_{\mathrm{Q}}(E)=\mathrm{DOS}(E)e^{2}$, where
$\mathrm{DOS}(E)$ is the density of states (averaged by cell
area). The DOS value is calculated at the energy level with a charge
cutoff such that
$|\sigma(E)| = 5 \times 10^{13}\ e\cdot\mathrm{cm}^{-2}$, calculated by
the relation of accumulated charge $\sigma(E)$ at CB or VB:

\begin{equation}
  \label{eq:method-2}
  |\sigma(E)| = \left|\int_{E_{\mathrm{BE}}}^{E} \mathrm{DOS}(E')e dE' \right|
\end{equation}
where $E_{\mathrm{BE}}$ is the energy of the CB or VB band edge. 
The DOS's used for the calculation of $C_{\mathrm{Q}}$ 
for all 2D-materials using HSE06 hybrid functional is shown in Supplementary Figure 8-59. 

%\subsection{Additional information}

\subsubsection{Data Availability}

The data that support the findings of this study 
are available within the paper and its Supplementary Information.  

\subsubsection{Competing interests}
The Authors declare no conflict of interests.

%{\bf Supplementary Information} accompanies this paper at
%http://www.nature.com/naturecommunications 



\subsubsection{Acknowledgments}
We acknowledge valuable discussions with Fritz B. Prinz, and Andres Castellanos-Gomez. C.J.S. and T.T. are grateful for financial support from ETH startup funding. L.H.L. thanks the financial support from Australian Research Council (ARC) via Discovery Early Career Researcher Award (DE160100796). E.J.G.S. acknowledges the use of computational resources from the UK national high performance computing service (ARCHER) for which access was obtained via the UKCP consortium (EPSRC grant ref EP/K013564/1); the UK Materials and Molecular Modelling Hub for access to THOMAS supercluster, which is partially funded by EPSRC (EP/P020194/1). The Queen's Fellow Award through the grant number M8407MPH, the Enabling Fund (A5047TSL), and the Department for the Economy (USI 097) are also acknowledged.

\subsubsection{Author Contributions}
E.J.G.S. designed the research and suggested the 2D-Moss' relation. 
D.H. and E.J.G.S. performed the first-principles simulations 
and data analytics. T.T. and C.J.S. developed the capacitor model. L.H.L. and J.C. 
performed numerical analysis and contributed to the discussions together with M.C.
E.J.G.S. and T.T. co-wrote the manuscript with inputs from all authors. 
All authors contributed to this work, read the manuscript, discussed 
the results, and all agree to the contents of the manuscript. 

\bibliography{refv030318}

%\clearpage 
\pagebreak{}

%\section{Figures}



\label{sec:org01da3bf}

\begin{figure}[htbp]
\centering
\includegraphics[width=0.91\linewidth]{fig1_v122017_ink.pdf}
\caption{\label{fig-1} {\bf Schematic of the studied 2D systems 
and their separation in power- and linear-law materials.}
{\bf a}, Schematic of the geometries of the 2D materials included in our database. 
From dichalcogenides at the most common phases, such as  
2H (P\(\bar{6}\)m2 space group) and 1T (P3m1 space group), up to different graphene derivatives, phosphorene, h-BN, CdCl$_2$ and popular organic-inorganic perovskites (CH$_3$NH$_3$PbBr$_3$). 
{\bf b}, Periodic table showing the main elements utilized in our high-throughput screening, and the classification of the groups accordingly to their scaling relationships. 
}
\end{figure}


\begin{figure}[htbp]
\centering
\includegraphics[width=0.67\linewidth]{fig1_v121217dielectric_xx_zz_ink.pdf}
\caption{\label{fig-2} {\bf 2D Moss relation between dielectric constants and band gaps.} 
{\bf a-b}, Calculated in-plane $\varepsilon_{\parallel}$ and out-of-plane $\varepsilon_{\perp}$  
dielectric constants for monolayer vdW materials as a function of the electronic band gap, respectively. 
The 2D materials are categorized into the linear group (green circles) and the 
power group (orange squares). Experimental values of dielectric 
constants and band gaps are show by 
blue squares for comparison\cite{BN-epsilon,Mos2-epsilon,Chernikov_2014_EB_MoS2_2D3D,In2Se3-epsilon}. 
Orange and green lines are fits to $\varepsilon_{\parallel}=3.68/E_{g}^{0.50}$ and 
$\varepsilon_{\parallel}=7.50-1.37E_{g}$, respectively, in {\bf a}. The original 3D-Moss relation, $\varepsilon_{\parallel}=9.74/E_{g}^{0.50}$, between the dielectric constant of bulk semiconductors and band gap is shown as faint blue line. The visible range between 1.7-3.3 eV is highlighted in both panels. 
Dataset for $\varepsilon_{\perp}$ in {\bf b} follows both straight lines, 
$\varepsilon_{\perp}=1.34-(0.00068)E_{g}$ (green) and $\varepsilon_{\perp}=1.28-(0.02)E_{g}$ (orange), 
with small dispersion around $\sim$1.30. The inset in {\bf b} shows the spatial definition of both $\varepsilon_{\perp}$ and $\varepsilon_{\parallel}$. All calculations are at hybrid functional HSE06 level. 
}
\end{figure}

\begin{figure}[htbp]
\centering
\includegraphics[width=\linewidth]{figure3_v011218ink.pdf}
\caption{\label{fig-3} {\bf Model prediction and a photoconductivity descriptor.}
{\bf a,b}, Comparison of HSE06-calculated \(\varepsilon_{\parallel}\) and \(\varepsilon_{\perp}\) 
and model predictions, respectively. The scaling relationships of in-plane ($\varepsilon_{\parallel}=3.68/E_{g}^{0.50}$ and $\varepsilon_{\parallel}=7.50-1.37E_{g}$) and out of plane ($\varepsilon_{\perp}=1.34-(0.00068)E_{g}$ and $\varepsilon_{\perp}=1.28-(0.02)E_{g}$) were utilized with $E_{g}$ as the only external parameter.  
{\bf c}, Calculated $n_{0}^{4}/\lambda_{c}$ for all materials considered in power- and linear-law (inset). The dashed blue and violet (in the inset) lines shows the average over the entire set of materials at each group. 
The back dashed line corresponds to Eq.\ref{Moss-2D-power-lambda}. 
Materials that have shown photoconductivity experimentally are marked with the red star\cite{Andres15,photo-2Dmaterials}. 
All calculations are using HSE06 hybrid functional. 
}
\end{figure}




\begin{figure}[htbp]
\centering
\includegraphics[width=0.95\linewidth]{fig4_121917_ink.pdf}
\caption{\label{fig-5}
{\bf The relation between \(\varepsilon_{\parallel}\) and \(C_{\mathrm{Q}}\) in the CB.} {\bf a} \(\varepsilon_{\parallel}\) as a function of \(C_{\mathrm{Q}}\), and {\bf b}, \(\varepsilon_{\parallel}\) as a function of \(C_{\mathrm{Q}}\)/$E_{g}^{2}$ for the 2D materials studied. Linear fitting is performed for materials of the metal element group and lattice type. The prediction margin of each group is shown in shader. {\bf c}, Scheme of the DOS obtained for each material groups highlighting the main stater near the Fermi level $E_F$. {\bf d}, Calculated DOS/$E_{g}^{2}$ using HSE06 hybrid functional for sample layers from group 4 (2H-ZrS$_2$, 2H-HfS$_2$) and group 6 (2H-MoS$_2$, 2H-WS$_2$). Valence band states (VB) and conduction band states (CB) are marked in faint red and green.
}
\end{figure}





%%%%%%% Only figures %%%%%%%%%%%

%\label{sec:org01da3bf}

%\setcounter{figure}{0}
%\begin{figure}[htbp]
%\centering
%\includegraphics[width=1.0\linewidth]{fig1_v122017_ink.pdf}
%\caption{\label{fig-1}
%}
%\end{figure}


%%\setcounter{figure}{0}
%\begin{figure}[htbp]
%\centering
%\includegraphics[width=0.850\linewidth]{fig1_v121217dielectric_xx_zz_ink.pdf}
%\caption{\label{fig-2}
% }
%\end{figure}

%\begin{figure}[htbp]
%\centering
%\includegraphics[width=1.1\linewidth]{figure3_v011218ink.pdf}
%\caption{\label{fig-3}
%}
%\end{figure}


%\begin{figure}[htbp]
%\centering
%\includegraphics[width=0.90\linewidth]{fig_biding_energy_v121917.pdf}
%\caption{\label{fig-4}
%}
%\end{figure}

%\begin{figure}[htbp]
%\centering
%\includegraphics[width=0.90\linewidth]{fig6_v122317_vertical.pdf}
%\caption{\label{fig-6}}
%\end{figure}


%\begin{figure}[htbp]
%\centering
%\includegraphics[width=1.1\linewidth]{fig4_121917_ink.pdf}
%\caption{\label{fig-5}
%}
%\end{figure}






\end{document}